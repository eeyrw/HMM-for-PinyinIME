\documentclass[12pt,a4paper,draft]{ctexart}
\usepackage[utf8]{inputenc}
\usepackage{amsmath}
\usepackage{amsfonts}
\usepackage{amssymb}
\usepackage{graphicx}
\usepackage{bm}
\usepackage[backend=biber,backref=true%nature,% citestyle=gb7714−2015,backref=true%
]{biblatex}
%参考文献数据源加载 
\addbibresource[location=local]{ref.bib}
%\usepackage[left=3.0cm, right=3.0cm, top=3.5cm, bottom=2.70cm]{geometry}
\title{隐马尔科夫模型 \\
	在拼音输入法中的应用研究}
\author{Yuan}
\date{\small\today}
\begin{document}

\maketitle
\begin{abstract}
本文就机器学习中常用的线性回归方法中经典的最小二乘法和用于处理大规模数据问题在线学习算法进行论述。
\end{abstract}	
\section{简介}
隐马尔科夫模型(Hidden Markov Model,HMM)是一种应用于序列问题的统计学系模型,该模型描述了由隐藏的马尔科夫链随机生成观测序列的过程。HMM非常适合解决标注问题,故HMM在语音识别、自然语言处理、生物信息、模式识别等领域有广泛的应用,本文着眼于将HMM应用于中文拼音输入法。拼音输入法将拼音转换为对应汉字,其中更具有实用价值的是将连续的拼音输入转换为连续的句子。这是典型的序列问题,更为具体地,是一种序列标注问题,故应用HMM能较好的解决该问题。
\section{问题描述}
HMM描述了一个隐藏的马尔科夫链生成不可观测的状态随机序列,再由各个状态生成一个观测而产生观测随机序列的过程。HMM随机生成的状态序列成为状态序列(state sequence);每一个状态生成一个观测,而由此产生的观测的随机序列成为观测序列(observation sequence)。故HMM可以由初始概率分布、状态转移概率分布以及观测概率分布确定,这三者对应的参数分别为初始状态概率向量$ \bm{\pi} $、状态转移概率矩阵$\bm{A}$及观测概率矩阵$\bm{B}$。由此HMM可以用三元组$ \lambda=(\bm{A},\bm{B},\bm{\pi}) $表示\cite{李航统计学习}。
\subsection{模型定义}
在拼音序列转汉字的问题中,连续的句子被视为序列,即认为每个时刻对应一个汉字,汉字又对应拼音。由于用户在输入法中输入的是拼音,所以拼音序列被定义为观测序列。而汉字则为生成观测序列的状态序列。故本应用下的HMM相关定义为:


所有的汉字组成集合Q(状态集合),所有汉字对应的拼音的集合V(观测集合),其中$ |Q|=N, |V|=M $。
\[ Q=\{ \mbox{汉字}_1,\mbox{汉字}_2,\cdots,\mbox{汉字}_N \},V=\{ \mbox{拼音}_1,\mbox{拼音}_2,\cdots,\mbox{拼音}_M \}  \]

用户想输入的汉字序列I(状态序列),用户实际输入的拼音序列O(观测序列),其中$ |I|=|O|=T $。
\[ I=\{ \mbox{预期汉字}_1,\mbox{预期汉字}_2,\cdots,\mbox{预期汉字}_T \}	\]
\[O=\{ \mbox{输入拼音}_1,\mbox{输入拼音}_2,\cdots,\mbox{输入拼音}_T \}  \]

句子中前后汉字转移概率矩阵$\bm{A}$:
\[ \bm{A}=[a_{ij}]_{N\times N} \]
其中,
\[ a_{ij}=P(i_{t+1}=\mbox{汉字}_j|i_t=\mbox{汉字}_i),  i=1,2,3,\cdots,N; j=1,2,\cdots,N\]
是在句子中的位置$ t $处的$ \mbox{汉字}_i $转移到$ t+1 $处的$ \mbox{汉字}_j $的概率。
句子中前后汉字转移概率矩阵$\bm{B}$:
\[ \bm{B}=[b_j(k)]_{N\times M} \]
其中,
\[ b_j(k)=P(\mbox{t处输入的拼音}=\mbox{拼音}_k|\mbox{ t处想要输入的汉字}=\mbox{汉字}_j) \]
\[ k=1,2,3,\cdots,M; j=1,2,\cdots,N\]
是在句子中的位置$ t $处的$ \mbox{汉字}_j $生成$ \mbox{拼音}_k $的概率。

句子中首个汉字出现可能性的概率向量$\bm{\pi}$:
\[ \bm{\pi}=(\pi_i) \]
其中,
\[ \pi_i=P(\mbox{句首汉字}=\mbox{汉字}_i), i=1,2,\cdots,N \]
是句首汉字为$\mbox{汉字}_i$的概率。
\subsection{模型训练(参数估计)}
HMM的训练可以采用监督学习或者无监督学习\cite{李航统计学习},本应用的状态和观测为汉字和拼音。将汉字转化为拼音可以自动开展,并有相当高的准确度\cite{accuracy-of-auto-pinyin}。故可以通过中文语料库生成用于监督学习的训练集。
本应用中HMM的三元组$ \lambda=(\bm{A},\bm{B},\bm{\pi}) $的估计方法如下:
\paragraph{前后汉字转移概率$a_{ij}$的估计}
\[ \hat{a}_{i j}=\frac{A_{ij}}{\sum_{j=1}^{N} A_{i j}}, \quad i=1,2, \cdots, N ; j=1,2, \cdots, N \]
\paragraph{汉字转拼音的概率$b_j(k)$的估计}
\[ \hat{b}_{j}(k)=\frac{B_{j k}}{\sum_{k=1}^{M} B_{j k}}, \quad j=1,2, \cdots, N ; k=1,2, \cdots, M \]
\paragraph{句首汉字出现概率$\pi_i$的估计}


\section{实验}
实验使用Souou实验室提供的往年新闻数据集作为语料,经过清洗后得到只含有的中文的句子片段\cite{viterbi2006a}
\printbibliography[heading=bibliography,title=参考文献]
\end{document}